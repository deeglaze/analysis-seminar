\documentclass[11pt]{article}
\usepackage{amsmath,amsthm,amssymb}
\newtheorem{theorem}{Theorem}
\newtheorem{definition}{Definition}
\newtheorem{lemma}[theorem]{Lemma}
\author{Dionna Glaze}
\title{Program Analysis Lecture Notes}
\begin{document}
\maketitle
\section{9 January 2013}

\subsection{Preliminaries}
Initial lectures will be about lattices and fixed points, classic
dataflow analysis (monotone), program dependency graphs and
nonstandard abstract semantics.

Prepare an annotated bibliography for papers you wish to study. All
participants will give a lecture on their chosen paper(s) and later a
short presentation on a project (implementation) that are related the these papers.

% Ian's reminder to self:
% SSA and sparse analysis (the first paper)
% A good sparse analysis (a value-flow paper on points-t analysis)
% Korean sparse analysis paper (general framework)

\subsection{Fixed points and lattices}

\begin{definition}[Preorder]
A relation $\le \subseteq P\times P$ that is reflexive and
transitive. The structure $(P,\le)$ is called a preordered set (preset).
\end{definition}
$P$ is called the \emph{carrier set}.

All structures in these definitions will be referred to by their
carrier set. Associated relations etc should be discernable from
context. If not, we disambiguate with a subscript.

\begin{definition}[Partial order]
An antisymmetric preorder. The structure $(P, \le)$ is called a partially ordered set (poset).
\end{definition}

\begin{definition}[Upper bound]
$a$ is an upper bound of $S \subseteq P$ in a poset
  $(P, \sqsubseteq)$ iff for all $x \in S$, $x \sqsubseteq a$.
\end{definition}

\begin{definition}[Supremum (least upper bound)]
$a$ is a supremum of $S \subseteq P$ in a poset $(P,
  \sqsubseteq)$ iff $a$ is an upper bound of $S$ and for all upper
  bounds $x$ of $S$, $a \sqsubseteq x$.
\end{definition}

Lower bound and greatest lower bound (infimum) are defined dually.

\begin{definition}[Lattice]
An extended poset $(L, \sqsubseteq, \sqcup, \sqcap)$ such that all
$a,b \in L$ have a supremum (join, $a \sqcup b$) and infimum (meet, $a
\sqcap b$).
\end{definition}

\begin{definition}[Complete Lattice]
A lattice $L$ such that all $A \subseteq
L$ have a supremum and infimum.
\end{definition}

\begin{definition}[Chain]
A sequence $a_0, a_1, ...$ in a poset $(P,
\sqsubseteq)$ such that $a_i \sqsubseteq a_{i+1}$ for all
$i$. (Another definition disallows repeating: totally ordered
non-empty subset)
\end{definition}

\begin{definition}[Finite height partial order]
All chains in the poset have a finite size (using subset definition).
\end{definition}

\begin{definition}[Pointed partial order]
A partial order where there are two special objects carried, $\top$ and $\bot$
such that for all $x \in P$, $\bot \sqsubseteq x \sqsubseteq \top$.
\end{definition}

Note that $sup(L) = \top$ and $inf(L) = \bot$ in a complete lattice.

\begin{definition}[Complete Partial Order (CPO)]
All chains have a supremum in the carrier set.
\end{definition}

\begin{definition}[Order-preserving function]
A function $f: X \to Y$, where $(X, \sqsubseteq)$ and $(Y, \preceq)$
are ordered sets, and for all $x,x' \in X$, if $x \sqsubseteq x'$ then
$f(x) \preceq f(x')$.
\end{definition}

\begin{definition}[Order-continuous function]
An order-preserving function $f : X \to Y$ between two complete
lattices $X$ and $Y$ such that for all $A \subseteq X$, $sup(f(A)) =
f(sup(A))$.
\end{definition}

\section{11 January 2013}
Out of town.

\end{document}
